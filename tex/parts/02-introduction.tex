\chapter*{Введение}
\addcontentsline{toc}{chapter}{Введение}

В современном мире стремительно растет объем доступной информации. В частности, огромное количество данных, в том числе неструктурированных, доступно в сети Интернет.  Так как значительная часть информации представлена в виде текстов на естественном языке, возникла необходимость анализа и обработки таких текстов. Одной из основных задач этого направления является обработка текстов в вопросно-ответных системах. В последнее время интерес исследователей смещается от традиционного поиска по запросу в сторону интеллектуальных систем поиска информации, поэтому существует множество методов, решающих задачу обработки текстов в вопросно-ответных системах.

\textbf{Целью данной работы} является классификация известных методов обработки текстов в вопросно-ответных системах.

Для достижения поставленной цели необходимо решить следующие задачи:
\begin{itemize}[label=---]
	\item провести обзор предметной области вопросно-ответного поиска;
	\item рассмотреть основные этапы работы вопросно-ответных систем;
	\item описать существующие методы обработки текстов, относящиеся к каждому из этапов;
	\item сформулировать критерии сравнения описанных методов;
	\item сравнить методы по сформулированным критериям. 
\end{itemize}