\chapter*{Заключение}
\addcontentsline{toc}{chapter}{Заключение}

В рамках научно-исследовательской работы была проведена классификация известных методов обработки текстов в вопросно-ответных системах.

В результате сравнения методов были выделены метод, использующий окна параграфа, для этапа информационного поиска, метод извлечения ответа с использованием N-грамм и метод сопоставления сказуемых для этапа валидации ответа, как наиболее точные из представленных. Для этапа анализа вопроса было определено целесообразным последовательное использование трех методов анализа.

В итоге, в ходе данной работы решены все задачи:
\begin{itemize}
	\item проведен обзор предметной области вопросно-ответного поиска;
	\item рассмотрены основные этапы работы вопросно-ответных систем;
	\item описаны существующие методы обработки текстов, относящиеся к каждому из этапов;
	\item сформулированы критерии оценки сравнения описанных методов;
	\item проведено сравнение методов по сформулированным критериям.
\end{itemize}

Таким образом, поставленная цель достигнута.



