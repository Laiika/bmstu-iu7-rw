\chapter{Классификация существующих решений}

В данном разделе предлагаются критерии оценки методов и проводится сравнение по выделенным критериям.

\section{Сравнение и оценка для этапа анализа вопроса}

Для оценки методов данного этапа выбраны такие критерии, как сложность реализации, влияющий на скорость разработки модуля анализа вопросов, покрытие вопросов методом и необходимость предварительной обработки вопроса. 

Результаты сравнения методов анализа вопроса приведены в таблице~\ref{tb:s1}. Оценки сложности реализации и покрытия даны по~\cite{step1}.

\begin{table}[h!]
	\begin{center}
		\begin{threeparttable}
			\captionsetup{justification=raggedright,singlelinecheck=off}
			\caption{\label{tb:s1} Сравнение методов анализа вопроса}
			\begin{tabular}{|p{3.5cm}|p{2.5cm}|p{2.5cm}|p{3.5cm}|}
				\hline
				Метод & Сложность\par реализации & Покрытие\par вопросов & Необходимость\par преварительной обработки\par вопроса \\ [0.8ex] 
				\hline
				Символьные\par шаблоны & Низкая & Низкое & Нет \\
				\hline
				Синтаксические\par шаблоны & Средняя & Высокое & Есть \\
				\hline
				Статистика\par употребления слов & Высокая & Среднее & Нет \\
				\hline
			\end{tabular}
		\end{threeparttable} 
	\end{center}
\end{table}

Метод символьных шаблонов обладает излишней простотой, поэтому корректно работает лишь в ограниченном числе случаев. Синтаксические шаблоны способны покрыть значительную часть реальных вопросов, но их сложнее подготовить, так же требуется построить синтаксическое дерево вопроса. Для корректной работы статистического метода необходимо создание большой обучающей коллекции вопросов вручную.

Представляется целесообразным последовательное использование всех методов. Символьные шаблоны будут эффективны на первом этапе обработки, и в случае полного соответствия вопроса шаблону обработка прекращается. В противном случае подключается имеющаяся статистика, и вопрос анализируется согласно ей. Если же для текущего вопроса статистика не собрана или недостаточно достоверна, применяются наиболее общие синтаксические шаблоны. 

\section{Сравнение и оценка для этапа информационного поиска}

Для оценки методов данного этапа выбраны такие критерии, как учет методом количества ключевых слов в текстовом фрагменте, учет порядка следования ключевых слов и учет расстояния между ключевыми словами~\cite{all}. 

Результаты сравнения методов приведены в таблице~\ref{tb:s2}.

\begin{table}[h!]
	\begin{center}
		\begin{threeparttable}
			\captionsetup{justification=raggedright,singlelinecheck=off}
			\caption{\label{tb:s2} Сравнение методов на этапе информационного поиска}
			\begin{tabular}{|p{4.1cm}|p{2.5cm}|p{2.5cm}|p{3.0cm}|}
				\hline
				Метод & Учет числа ключевых слов & Учет порядка ключевых слов & Учет расстояния между словами\\ [0.8ex] 
				\hline
				Деление на абзацы & Есть & Нет & Нет \\
				\hline
				Использование окон параграфа & Есть & Есть & Есть \\
				\hline
			\end{tabular}
		\end{threeparttable} 
	\end{center}
\end{table}

По данным таблицы~\ref{tb:s2} наиболее точным является метод, использующий окна параграфа, так как он учитывает не только количество ключевых слов в текстовом фрагменте, но и их расположение.



\section{Сравнение и оценка для этапа извлечения потенциальных ответов}

Для оценки методов используются следующие метрики:

\begin{itemize}[label=---]
	\item среднеобратный ранг (MRR) --- оценка извлеченных ответов, упорядоченных по вероятности и правильности;
	\item доля вопросов, на которые были даны правильные ответы.
\end{itemize}

Результаты сравнения методов извлечения потенциальных ответов приведены в таблице~\ref{tb:s3}.

\begin{table}[h!]
	\begin{center}
		\begin{threeparttable}
			\captionsetup{justification=raggedright,singlelinecheck=off}
			\caption{\label{tb:s3} Сравнение методов извлечения потенциальных ответов}
			\begin{tabular}{|c|c|c|}
				\hline
				Метод & MRR & Доля вопросов \\ [0.8ex] 
				\hline
				Использование шаблонов~\cite{patterns2} & 0.29 & 0.25 \\
				\hline
				Использование N-грамм~\cite{ns2} & 0.42 & 0.49 \\
				\hline
			\end{tabular}
		\end{threeparttable} 
	\end{center}
\end{table}

По данным таблицы~\ref{tb:s3} наиболее точным является метод извлечения ответа с использованием N-грамм, однако важно отметить, что значения в первой строке таблицы~\ref{tb:s3} сильно зависят от полноты шаблонов и могут отличаться в различных работах.

\section{Сравнение и оценка для этапа валидации ответов}

Данный способ оценки валидации ответов основан на традиционном подходе к оценке в задаче классификации~\cite{valid}. Задача валидации рассматривается как задача бинарной классификации: тройку <вопрос, ответ, текстовый фрагмент> требуется отнести к одному из классов --- верный ответ или неверный.

В таблице~\ref{tb:s4} приведены четыре возможных исхода решения задачи классификации.

\begin{table}[!h]
	\begin{center}
		\begin{threeparttable}
			\captionsetup{justification=raggedright,singlelinecheck=off}
			\caption{\label{tb:s4}Категории результата классификации ответов}
			\begin{tabular}{|c|c|c|}
				\hline
				\multirow{2}{*}{Наблюдаемый результат}&\multicolumn{2}{|c|}{Ожидаемый результат}\\
				{} & Верный ответ & Неверный ответ\\
				\hline
				Верный ответ & tp (true-positive) & fp (false-positive)\\ 
				\hline
				Неверный ответ & fn (false-negative) & tn (true-negative)\\
				\hline
			\end{tabular}
		\end{threeparttable} 
	\end{center}
\end{table}

На основе этой таблицы определяются традиционные метрики качества классификации:

\textbf{Accuracy} --- доля ответов, по которым классификатор принял правильное решение. Вычисляется по формуле~(\ref{eq:31}):

\begin{equation}\label{eq:31}
	Accuracy = \frac{tp + tn}{tp + tn + fp + fn}.
\end{equation}

\textbf{Точность (precision)} --- доля ответов, действительно принадлежащих данному классу, относительно всех ответов, которые система отнесла к этому классу. Вычисляется по формуле~(\ref{eq:32}):

\begin{equation}\label{eq:32}
	Precision = \frac{tp}{tp + fp}.
\end{equation}

\textbf{Полнота (recall)} --- доля ответов, причисленных классификатором к данному классу, относительно всех ответов, принадлежащих ему в тестовой выборке. Вычисляется по формуле~(\ref{eq:33}):

\begin{equation}\label{eq:33}
	Recall = \frac{tp}{tp + fn}.
\end{equation}

\textbf{F-мера} --- среднее гармоническое точности и полноты, вычисляется по формуле~(\ref{eq:34}):

\begin{equation}\label{eq:34}
	F_{\beta} = \frac{(1 + \beta^2) \cdot Precision \cdot Recall}{\beta^2 \cdot Precision + Recall},
\end{equation}

\noindent где коэффициент $\beta$ может рассматриваться как относительная степень важности показателей полноты и точности. При значении коэффициента равном 1/2 точность вдвое важнее полноты, при значении равном 2 полнота вдвое важнее точности.

Результаты сравнения методов валидации ответов приведены в таблице~\ref{tb:s42}. Значения метрик даны по~\cite{valid}.

\begin{table}[h!]
	\begin{center}
		\begin{threeparttable}
			\captionsetup{justification=raggedright,singlelinecheck=off}
			\caption{\label{tb:s42} Сравнение методов валидации ответов}
			\begin{tabular}{|c|c|c|c|c|}
				\hline
				Метод & $Accuracy$ & $Precision$ & $Recall$ & $F_{0.5}$ \\ [0.8ex] 
				\hline
				Пересечение множеств слов & 0.62 & 0.51 & 0.47 & 0.48 \\
				\hline
				Сопоставление сказуемых & 0.70 & 0.72 & 0.27 & 0.54 \\
				\hline
				Расстояние редактирования & 0.64 & 0.45 & 0.09 & 0.26 \\
				\hline
			\end{tabular}
		\end{threeparttable} 
	\end{center}
\end{table}

По данным таблицы~\ref{tb:s42}, наиболее точным является метод сопоставления сказуемых. Хотя метод пересечения множеств слов имеет лучший показатель полноты, на данном этапе показатель точности является более важным.

