\chapter{Анализ предметной области}

\section{Вопросно-ответный поиск}
Современные поисковые системы выдают информацию в виде документов, веб-страниц, изображений и видео по запросу, но распознавать запросы на естественном языке и формировать ответы в соответствующем формате большая часть из них не способна. В результате подобных запросов пользователь получается множество ссылок на различные сайты и документы, которые ему необходимо изучить, чтобы найти ответ на свой вопрос. От обычных поисковых систем выгодно отличаются системы вопросно-ответного поиска.

Вопросно-ответные системы --- это вид информационно-поисковых систем, способных обрабатывать введенный пользователем вопрос на естественном языке и выдавать осмысленный ответ~\cite{class1}.

В системах вопросно-ответного поиска активно используются технологии обработки естественного языка. Сначала на вход системе подаётся запрос, сформулированный в виде вопросительного предложения на естественном языке. Далее этот запрос обрабатывается, происходит поиск и вывод ответа в виде одного или нескольких слов на естественном языке, либо небольшого фрагмента текста, сформированного системой в результате анализа и обработки разнообразных источников данных. Источником информации для поиска ответа может быть как Интернет, так и локальное хранилище данных. Вопросно-ответные системы отличаются тематикой вопросов, которые они могут обрабатывать.

\section{Виды вопросно-ответных систем}
Системы вопросно-ответного поиска можно разделить на две группы~\cite{art1}: 
\begin{enumerate}
	\item системы без специализации тематики;
	\item узкоспециализированные по тематике системы.
\end{enumerate}

Первый вид вопросно-ответных систем ориентирован на обработку вопросов по любым предметным областям. В качестве источника информации общие системы используют большой корпус документов или сеть Интернет.
Узкоспециализированные системы направлены на ответы на вопросы по конкретным предметным областям, например, медицина, юриспруденция. Релевантность и полнота ответов зависят от полноты знаний о домене, которому посвящён диалог.

Существуют различные принципы построения вопросно-ответных систем, но основными являются следующие~\cite{class1}: 
\begin{enumerate}
	\item метапоисковые вопросно-ответные системы;
	\item вопросно-ответные системы поиска ответа по аннотированному тексту;
	\item вопросно-ответные системы поиска ответа в коллекциях вопросов и ответов;
	\item экспертные вопросно-ответные системы.
\end{enumerate}

\subsection{Метапоисковые системы}

В качестве источника данных такая система использует классическую поисковую систему, то есть использует неструктурированные данные.

Вопросно-ответная система после получения на вход от пользователя вопросительного предложения на естественном языке обрабатывает это предложение и формирует запрос для поисковой системы из ключевых слов, которые выбираются исходя из самого вопросительного предложения. Результаты поиска обрабатываются существующими компонентами систем автоматической обработки текста. Например, выделяются все именованные сущности, соответствующие искомому классу ответа: персоны, географические названия, названия организаций, линейные размеры и другие. Далее синтаксический и семантический разбор позволяют выбрать из всех найденных сущностей наиболее подходящие.

\subsection{Системы поиска ответа по аннотированному тексту}

Такие системы имеют в своем составе поисковый индекс документов и работают с неструктурированными данными. Элементами индекса являются не отдельные слова текста, а объекты детального лингвистического анализа: именованные сущности, элементарные синтаксические связки (пары грамматически связанных слов и другие). Построение индекса происходит с привлечением компьютерной лингвистики, а именно каждый новый документ проходит этапы автоматической обработки текста на естественном языке, размечаются объекты вопросно-ответной системы (именованные сущности, элементарные синтаксические связки), затем они добавляются в индекс.   

\subsection{Системы поиска ответа в коллекциях вопросов и ответов}

В социальных системах вопросно-ответного поиска (англ. \textit{collaborative question answering}) одни пользователи отвечают на вопросы других. Пользователь открывает страницу веб-сайта и формулирует вопрос. Система ищет похожие вопросы в коллекции вопросов и ответов и выдает найденный раздел, где обсуждается вопрос. Если подобный вопрос не существует, создается новый раздел для обсуждения вопроса. На этот вопрос отвечают желающие, а автору приходят уведомления по мере появления ответов. Данные в такой системе представлены в виде коллекции вопросов с ответами.

\subsection{Экспертные системы}

Вопросно-ответные системы, построенные по принципу работы со структурированными базами
данных, можно отнести к классу экспертных систем. 

Основными компонентами экспертной системы являются база фактов и база правил. 
База фактов --- это структурированная база данных, которая может быть построена
автоматически в результате анализа коллекции документов. Этот процесс аналогичен
построению аннотированного индекса. Однако он происходит на более детальном уровне
обработки естественного текста: извлекаются не синтаксические конструкции, а факты. В базе правил хранятся процедуры для установления различных типов связей между фактами. Эти процедуры содержат информацию, позволяющую выполнять логический вывод новой информации на основе имеющихся фактов. Результатом является база знаний, позиционируемая как семантическая сеть.

Важным элементом экспертных систем также является некоторая управляющая структура, которая определяет --- какое из правил должно быть проверено следующим при формировании новой информации.

Такие системы являются узкоспециалированными из-за сложности организации базы фактов. Подобная база должна состоять только из достоверной информации о предметной области. 

\section{Этапы вопросно-ответного поиска}

Процесс работы вопросно-ответной системы можно разделить на следующие этапы~\cite{step21}: 
\begin{enumerate}
	\item этап анализа вопроса;
	\item этап информационного поиска;
	\item этап извлечения потенциальных ответов;
	\item этап валидации ответов.
\end{enumerate}

\subsection{Анализ вопроса}

На этапе анализа вопроса происходит ввод пользователем вопроса на естественном языке и дальнейшая его обработка. Вопросы можно разделить по виду ответа на следующие~\cite{all}: 
\begin{itemize}[label=---]
	\item фактографические вопросы;
	\item вопросы причины; 
	\item вопросы мнения. 
\end{itemize}

Фактографический вопрос --- это вопрос о различных сведениях без их анализа и обобщения,  ответ на данный вопрос обычно краток. Примерами фактографических вопросов являются вопросы о персонах, о времени, вопросы, требующие ответа <<да>> или <<нет>>.

Вопросы причины требуют логического анализа текста, определения между предложениями причинно-следственной связи и являются самыми сложными в плане нахождения ответа. Существующие решения в области анализа текста и вывода его логической структуры довольно плохо справляются с данным видом вопросов. 

Вопросы мнения предусматривают собой поиск и глубокий анализ блогов и различных сайтов СМИ.

На этапе анализа вопроса ставится следующая задача: для вопроса на естественном языке выделить фокус вопроса, опору вопроса и определить семантический тэг ответа \cite{step1}.

\textbf{Фокус вопроса} --- это такие сведения, содержащиеся в вопросе, которые несут в себе информацию об ожиданиях пользователя от информации в ответе, т.~е. вопросительные слова,
обозначающие искомую информацию, например, <<в каком городе>>, <<кто>>, <<в каком году>>,  <<сколько>>,  <<какого цвета>> и другие.

\textbf{Опора вопроса} --- это остальная часть вопроса (после <<вычета>> фокуса), которая несёт в себе информацию, поддерживающую выбор конкретного ответа. В опору вопроса входят ключевые слова, по которым система формулирует запрос для информационного поиска.

\textbf{Семантический тэг ответа} --- класс вопроса, соответствующий типу возможного ответа. Многие вопросно-ответные системы имеют встроенную систему поддерживаемых типов ожидаемого ответа, иногда используется иерархическая структура представления в виде таксономии в зависимости от дальнейшей стратегии поиска и извлечения ответа.

\subsection{Информационный поиск}

На этом этапе производится поиск релевантных запросу документов, а также получение текстовых фрагментов, содержащих ответ~\cite{step21}. Результат должен содержать необработанный или в редких случаях готовый ответ на вопросительное предложение пользователя.

\subsection{Извлечение потенциальных ответов}

На данном этапе распознаются и извлекаются из полученных текстовых фрагментов потенциальные ответы на вопрос. Важную роль в выделении ответа играет тип ответа. 

\subsection{Валидация ответов}

На последнем этапе производится анализ списка кандидатов, все кандидаты оцениваются и выбирается наиболее подходящий.
